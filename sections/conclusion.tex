\chapter{Conclusion}
\label{cha:conclusion}

6LoCAN brings end-to-end IPv6 to small-scale microcontrollers with only a few additional components, like the CAN transceiver.
It can be used for non-realtime communication and bulky data transfers.
If configured to use a random address assignment, it is zero-configuration capable and does not require any settings or persistent memory.
Nodes can join a 6LoCAN network seamlessly. The nodes could, for example, be identified with host-names and multicast DNS discovery.
The CAN-bus is not the most efficient bearer for IPv6 traffic, because of the small frame payload of eight bytes for classic CAN, but with the 64-byte payload of CAN-FD, packets with small payload can even fit a single frame.
By using IPv6 instead of raw data transfers, it is possible to use any application-layer protocol that works on top of IP.
With IPv6, it is, for example, possible to encrypt the traffic using the well known TLS protocol \cite{rfc8446}.
Devices that already have native IPv6 support can easily be connected to large-scale networks like the internet.
6LoCAN supports multicast groups natively, and therefore, very efficiently.
Packets to dedicated groups are only received by nodes that subscribed to the group.
This property could be used for efficient implementations of a publisher and subscriber models like MQTT.
Since 6LoCAN only uses the 29-bit extended addresses, the 11-bit standard address range is still usable for other protocols.
An 11-bit standard address always has a higher priority on the bus than an extended address, and therefore the standard address-range can still even be used for high priority traffic, like real-time events.
It could also share the bus with other existing protocols like CANopen.
CANopen could, for example, be used for machine control, as ith has been used before, and 6LoCAN could be used for interfacing control panels.

The 6LoCAN standard-draft \cite{wachter-6lo-can-00} was submitted to the Internet Engineering Task Force (IETF) in October 2019.