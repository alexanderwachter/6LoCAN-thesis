\chapter{Implementation}
\label{cha:implementation}

The 6LoCAN protocol is implemented into the Zephyr RTOS and made its way to the 2.0 release.
The implementation was made in two phases.
First we introduced a generic CAN API and an implementation of it for the STM32 microcontroller platform.
In the second phase, I implemented the 6LoCAN translation layer into the Zephyr networking stack.

\section{The Zephyr CAN API}
\label{sec:can_api}
Before this work, the Zephyr RTOS was lacking a CAN API, so the first step was to create a platform-agnostic API.
The API should only support features defined by the CAN specification \cite{BoschCAN},
to ensure that the network or application layer can rely on the API features.

The first API had an interface for the following features:
\begin{itemize}
        \item Configure the mode and bitrate of the controller.
        \item Send a frame
        \item Attach a CAN filter and submit a callback whenever the filter matches on a frame
        \item Attach a CAN filter and submit the frame to a message queue whenever the filter matches
        \item Detach attached filters
\end{itemize}

It also defined containers for CAN frames and CAN filters called zcan\_frame and zcan\_filter.
The containers provide a platform-agnostic way to handle frames and filters.
The specific driver implementation is responsible for converting the zcan\_frame and zcan\_filter to the respective register values.

The Zephyr community agreed on my API proposal and the implementation for the STM32 platform and it was merged merged into Zephyr version 1.12.
The API evolved over the time and got extended by an API the get the bus state and recover from bus-off state.
Additionally people from the Zephyr community added two other driver implementations.
One for an external CAN controller (MCP2515) and one for NXP platforms, based on NXPs flexcan IP.

\subsection{Zephyr CAN frame}
\begin{lstlisting}[style=ccode, numbers=none]
struct zcan_frame {
	u32_t id_type : 1;
	u32_t rtr     : 1;
	u32_t ext_id  : 29;
	u8_t  dlc;
	u8_t  data[8];
};
\end{lstlisting}
The struct above shows a simplified version of a Zephyr CAN frame.
It is a container for a CAN frame and includes the identifier type, the identifier, a flag for RTR frames, the data length code, and the data data.

\subsection{Zephyr CAN filter}
\begin{lstlisting}[style=ccode, numbers=none]
struct zcan_filter {
	u32_t id_type  : 1;
	u32_t rtr      : 1;
	u32_t ext_id   : 29;

	u32_t rtr_mask    : 1;
	u32_t ext_id_mask : 29;
}
\end{lstlisting}
The struct above shows a simplified version of a Zephyr CAN filter struct.
It is a container for a CAN filter and includes the identifier type, the identifier, the RTR bit, and the masks for the identifier and RTR-bit.
The mask signals which bits of the identifier should be taken into account during the filtering.
Bits that are set to one are compared and bits that are zero are ignored.
Lets take an example of a filter with an identifier of 0x123 and a mask of 0xff.
Frames with the identifier 0x123, 0x223, or generally 0xX23, where X is any number, would pass the filter.
Frames with the identifier 0x124, 0x133, or generally 0xYXX, where XX is not 0x23, will not pass the filter.

\subsection{Sending CAN frames}
\begin{lstlisting}[style=ccode, numbers=none]
int can_send(struct device *dev, const struct zcan_frame *msg,
             s32_t timeout, can_tx_callback_t callback_isr,
             void *callback_arg);
\end{lstlisting}
The Send API is defined as shown above.
It takes a pointer to the CAN device, a pointer to the frame container, a timeout, a callback function and an argument that is passed to the callback function.
The function sends the frame as soon as a mailbox is ready to send a message.
If a NULL pointer is passed to as the callback function, the call blocks until the message is sent.
Otherwise the callback is called when the message is sent.

\subsection{Receiving CAN frames}

\begin{lstlisting}[style=ccode, numbers=none]
int can_attach_isr(struct device *dev,
                   can_rx_callback_t isr, void *callback_arg,
                   const struct zcan_filter *filter);
\end{lstlisting}
The main receiving function is shown above.
It takes a pointer to the CAN device, the function that should be called when a frame that passes the filter is received,
an argument that is passed to the callback and the CAN filter.
The function returns a filter-id to uniquely identify the attached filter.
The filter-id is used for detaching the filter when needed.
The CAN controller reads any message on the bus and compares the identifier to the attached filters.
If a filter matches, the respective callback is called.
If the identifier matches more that one filter, the function called depends on the metrics of the CAN controller.
The callback is called in an interrupt context and therefor is not allowed to block and should keep the execution time as low as possible.

\begin{lstlisting}[style=ccode, numbers=none]
int can_attach_workq(struct device *dev, struct k_work_q *work_q,
                     struct zcan_work *work,
                     can_rx_callback_t callback,
                     void *callback_arg,
                     const struct zcan_filter *filter);
\end{lstlisting}
The can\_attach\_workq function is a wrapper for the can\_attach\_isr,
that calls the callback function in the context of the provided work queue instead of an ISR context.

\begin{lstlisting}[style=ccode, numbers=none]
int can_attach_msgq(struct device *dev, struct k_msgq *msg_q,
                    const struct zcan_filter *filter);
\end{lstlisting}
The can\_attach\_msgq function attaches a message queue instead on a callback.
Whenever a frame that matches the filter is received, it is put into the message queue.
The message queue can then be read within a thread.

\section{6LoCAN Implementation}
\label{sec:6locan_impl}

\begin{figure}[htp]
	\begin{center}
	\begin{tikzpicture}[
		scale=0.7,
		every node/.append style={scale=0.7},
		layers/.style={
			align=left,
			fill=blue!5,
			rectangle,
			draw=blue!5,
			minimum width = 12cm,
			node distance=0.2cm},
		layers_dark/.style={
			draw=black,
			fill=blue!15,
			rectangle,
			minimum width = 12cm,
			minimum height = 1cm,
			node distance=0.2cm},
		layer_rot/.style={
			draw=black,
			fill=blue!15,
			rectangle},
		instance/.style={
			draw=black,
			rounded corners=0.05cm,
			fill=blue!15,
			rectangle, 
			minimum width=2cm,
			minimum height=1cm,
			align=center,
			font=\small},
		instance_wide/.style={
			draw=black,
			rounded corners=0.05cm,
			fill=blue!15,
			rectangle, 
			minimum width=4cm,
			minimum height=1cm,
			align=center,
			font=\small}
		]

		%App
		\node (app)[layers_dark] {Networking Application};
	
	
	
		%Sockets API
		\node (sockets)[layers_dark, below = of app.south west,anchor=north west] {Sockets API};

		%Net-context
		\node (netctx)[layers_dark, below = of sockets.south west,anchor=north west] {Net-Context API};
	
		%Transport-Layer Protocols
		\node (transp)[layers,below = of netctx.south west,anchor=north west, text width = 9.1cm, minimum height = 4cm, text depth = 3cm] {Transport-Layer Protocols};
		\foreach \tansport/\x/\y in {ICMPv4/-2.5/0.75, ICMPv6/2.5/0.75, TCP/-2.5/-0.5, UDP/2.5/-0.5}{
			\node (\tansport) at ($(transp) + (\x cm, \y cm -0.5 cm)$) [instance_wide] {\tansport};
		}

		%Networking Protocols
		\node (netw)[layers,below = of transp.south west,anchor=north west, text width=9.1cm, minimum height = 2.5cm, text depth = 1.5cm] {Networking Protocols};
		\foreach \network/\x in {IPv4/-2.5, IPv6/2.5}{
			\node (\network) at ($(netw) + (\x cm, -0.5 cm)$) [instance_wide] {\network};
		}

		%Interface Abstraction
		\node (netcore)[layers_dark, below = of netw.south west,anchor=north west] {Network Core};

		%Interface Abstraction
		\node (interf)[layers_dark, below = of netcore.south west,anchor=north west] {Network Interface Abstraction Layer};

		%Link-Layer
		\node (linklayer)[layers, below = of interf.south west,anchor=north west, text width = 10.3cm, minimum height = 2.5cm, text depth = 1.5cm] {Link-Layer Technologies};
		\foreach \linklayer/\x/\col in {Ethernet/-4/blue!15, 6LoWPAN/-1.25/blue!15, 6LoCAN/1.25/green!15, IPSP/4/blue!15}{
			\node (\linklayer) at ($(linklayer) + (\x cm, -0.5 cm)$) [instance, fill=\col] {\linklayer};
		}

		%Network drivers
		\node (net_drivers)[layers, below = of linklayer.south west,anchor=north west, text width = 11.5cm, minimum height = 3cm, text depth = 2cm] {Network Device Drivers};
		\foreach \driver/\x/\col in {Ethernet/-4.5/blue!15, CAN/-1.5/green!15, 802.15.4/1.5/blue!15, Other/4.5/blue!15}{
			\foreach \inst in {0.3,0.2,0.1}{
				\node at ($(net_drivers) + (\x cm - \inst cm, -0.5 cm + \inst cm)$) [instance] {};
			}
			\node (\driver)  at ($(net_drivers) + (\x cm, -0.5 cm)$) [instance, fill=\col] {\driver \\ drivers};
		}

		\draw [arrow] (CAN.north) -- node[yshift=-1.5cm,draw, circle, anchor=west, xshift=0.1cm]{1} (netcore.south -| CAN.north);
		\draw [arrow] ([xshift=1cm]netcore.south) coordinate (netcore1) -- node[draw, circle, anchor=east, xshift=-0.1cm]{2} (interf.north -| netcore1);
		\draw [arrow_rev] ([xshift=1.5cm]netcore.south) coordinate (netcore2) -- node[draw, circle, anchor=west, xshift=0.1cm]{5} (interf.north -| netcore2);
		\draw [arrow_rev] ([xshift=-0.25cm]6LoCAN.north) coordinate (6locan1) --  node[draw, circle, anchor=east, xshift=-0.1cm]{3} (interf.south -| 6locan1);
		\draw [arrow] ([xshift= 0.25cm]6LoCAN.north) coordinate (6locan2) -- node[draw, circle, anchor=west, xshift=0.1cm] {4} (interf.south -| 6locan2);
		\draw [arrow_rev] (IPv6.south) -- node[draw, circle, anchor=west, xshift=0.1cm] {6} (netcore.north -| IPv6.south);
		\draw [arrow] ([xshift=0.25cm]IPv6.north) coordinate (udp1) -- node[draw, circle, anchor=west, xshift=0.1cm] {7} (UDP.south -| udp1);
		\draw [arrow_rev] ([xshift=-0.25cm]IPv6.north) coordinate (udp2) -- node[draw, circle, anchor=east, xshift=-0.1cm] {8} (UDP.south -| udp2);
		\draw [arrow] ([xshift=-1cm]IPv6.north) -- node[draw, circle, anchor=east, xshift=-0.1cm, yshift=-0.1cm] {9} (netctx);
		\draw [arrow] (netctx) -- node[draw, circle, anchor=east, xshift=-0.1cm] {10} (sockets);
		\draw [arrow] (sockets.north) -- node[draw, circle, anchor=east, xshift=-0.1cm] {11} (app.south);
	\end{tikzpicture}
	\caption{Zephyr Network Stack 6LoCAN RX example}
	\label{fig:zephyr_net_stack_rx}
	\end{center}
\end{figure}


A Link-Layer Implementation in Zephyr consist of two layers.
The Network Device Driver and the Link-Layer implementation (L2), which are highlighted in \autoref{fig:zephyr_net_stack_rx}.
For the 6LoCAN implementation, the Network Device Driver is an abstraction of the Zephyr CAN API that installs the respective frame filters,
handles the incoming CAN frames in an interrupt context, puts them into network packets, sets the Link-Layer address of them,
and and hands the packet over to the receiving work-queue. For sending CAN frames, the abstraction only forwards the raw CAN frames.
The 6LoCAN Link-Layer implementation does the parsing of the ISO-TP header, fragmentation and reassembly, and IPHC header compression and decompression.
The device driver implementation is explained in detail in \autoref{sec:6locan_net_dev},
and the L2 implementation is explained in detail in \autoref{sec:6locan_l2}.

\newpage

\autoref{fig:zephyr_net_stack_rx} shows an example of receiving a UDP packet.
\begin{enumerate}
        \item The CAN network device driver uses the CAN API to receive the raw CAN frames.
              The frames are copied into network packets and handed over to the Net Core.
              The Net Core accepts the packet and puts it into a receiving work-queue, depending on the priority of the packet.
        \item When the receiving queue is scheduled, the packet is handed over to the Network Interface.
        \item The Network Interface calls the 6LoCAN L2 implementation with the received packet, which analyzes the frame content.
              The L2 implementation performs the reassembly (ISO-TP) and when the packet is complete, the IPHC decompression.
        \item If the packet is complete and decompression is successful, the reassembled packet is passed back to the Network Interface.
        \item The packet is then forwarded to the Net Core.
        \item The Network Core checks the IP header version and if it is IPv6, the IPv6 implementation is called.
              The IPv6 implementation checks the IPv6 headers and discards the packet if the destination does not match any addresses of the interface.
        \item Depending on the next header field, the specific transport layer implementation is called. In this Example, it is UDP.
        \item The UDP implementation parses the UDP header, checks the checksum and length of the packet, and returns the result to the IPv6 implementation.
        \item If the packet is valid, the IPv6 implementation passes the packet to the Net-Context API.
        \item The Net-Context API checks for registered sockets on the destination-port.
        \item If a socket is listening to the destination-port, the Socket-API passes the packet to the networking-application.
\end{enumerate}

\newpage
\begin{figure}[htp]
	\begin{center}
	\begin{tikzpicture}[
		scale=0.7,
		every node/.append style={scale=0.7},
		layers/.style={
			align=left,
			fill=blue!5,
			rectangle,
			draw=blue!5,
			minimum width = 12cm,
			node distance=0.2cm},
		layers_dark/.style={
			draw=black,
			fill=blue!15,
			rectangle,
			minimum width = 12cm,
			minimum height = 1cm,
			node distance=0.2cm},
		layer_rot/.style={
			draw=black,
			fill=blue!15,
			rectangle},
		instance/.style={
			draw=black,
			rounded corners=0.05cm,
			fill=blue!15,
			rectangle, 
			minimum width=2cm,
			minimum height=1cm,
			align=center,
			font=\small},
		instance_wide/.style={
			draw=black,
			rounded corners=0.05cm,
			fill=blue!15,
			rectangle, 
			minimum width=4cm,
			minimum height=1cm,
			align=center,
			font=\small}
		]

		%App
		\node (app)[layers_dark] {Networking Application};
	
	
	
		%Sockets API
		\node (sockets)[layers_dark, below = of app.south west,anchor=north west] {Sockets API};

		%Net-context
		\node (netctx)[layers_dark, below = of sockets.south west,anchor=north west] {Net-Context API};
	
		%Transport-Layer Protocols
		\node (transp)[layers,below = of netctx.south west,anchor=north west, text width = 9.1cm, minimum height = 4cm, text depth = 3cm] {Transport-Layer Protocols};
		\foreach \tansport/\x/\y in {ICMPv4/-2.5/0.75, ICMPv6/2.5/0.75, TCP/-2.5/-0.5, UDP/2.5/-0.5}{
			\node (\tansport) at ($(transp) + (\x cm, \y cm -0.5 cm)$) [instance_wide] {\tansport};
		}

		%Networking Protocols
		\node (netw)[layers,below = of transp.south west,anchor=north west, text width=9.1cm, minimum height = 2.5cm, text depth = 1.5cm] {Networking Protocols};
		\foreach \network/\x in {IPv4/-2.5, IPv6/2.5}{
			\node (\network) at ($(netw) + (\x cm, -0.5 cm)$) [instance_wide] {\network};
		}

		%Interface Abstraction
		\node (netcore)[layers_dark, below = of netw.south west,anchor=north west] {Network Core};

		%Interface Abstraction
		\node (interf)[layers_dark, below = of netcore.south west,anchor=north west] {Network Interface Abstraction Layer};

		%Link-Layer
		\node (linklayer)[layers, below = of interf.south west,anchor=north west, text width = 10.3cm, minimum height = 2.5cm, text depth = 1.5cm] {Link-Layer Technologies};
		\foreach \linklayer/\x/\col in {Ethernet/-4/blue!15, 6LoWPAN/-1.25/blue!15, 6LoCAN/1.25/green!15, IPSP/4/blue!15}{
			\node (\linklayer) at ($(linklayer) + (\x cm, -0.5 cm)$) [instance, fill=\col] {\linklayer};
		}

		%Network drivers
		\node (net_drivers)[layers, below = of linklayer.south west,anchor=north west, text width = 11.5cm, minimum height = 3cm, text depth = 2cm] {Network Device Drivers};
		\foreach \driver/\x/\col in {Ethernet/-4.5/blue!15, CAN/-1.5/green!15, 802.15.4/1.5/blue!15, Other/4.5/blue!15}{
			\foreach \inst in {0.3,0.2,0.1}{
				\node at ($(net_drivers) + (\x cm - \inst cm, -0.5 cm + \inst cm)$) [instance] {};
			}
			\node (\driver)  at ($(net_drivers) + (\x cm, -0.5 cm)$) [instance, fill=\col] {\driver \\ drivers};
		}

		\draw [arrow] (app.south) -- node[draw, circle, anchor=east, xshift=-0.1cm] {1} (sockets.north);
		\draw [arrow] (sockets) -- node[draw, circle, anchor=east, xshift=-0.1cm] {2} (netctx);
		\draw [arrow_both] ([xshift=0.5cm]IPv6.north) coordinate (IPv61) -- node[draw, circle, anchor=west, xshift=0.1cm, yshift=2cm] {3} (netctx.south -| IPv61);
		\draw [arrow_both] ([xshift=-0.5cm]UDP.north) coordinate (UDP1) -- node[draw, circle, anchor=east, xshift=-0.1cm, yshift=0.5cm] {4} (netctx.south -| UDP1);
		\draw [arrow] (netctx) -- node[draw, circle, anchor=east, xshift=-0.1cm, yshift=-0.5cm] {5} (netcore);
		\draw [arrow] (netcore) -- node[draw, circle, anchor=east, xshift=-0.1cm] {6} (interf);
		\draw [arrow_rev] (6LoCAN.north) coordinate (6locan1) --  node[draw, circle, anchor=east, xshift=-0.1cm]{7} (interf.south -| 6locan1);
		\draw [arrow] (6LoCAN.south) --  node[draw, circle, anchor=east, xshift=-0.1cm, yshift=0.1cm]{8} (CAN.north);

	\end{tikzpicture}
	\caption{Zephyr Network Stack 6LoCAN TX example}
	\label{fig:zephyr_net_stack_tx}
	\end{center}
\end{figure}


\autoref{fig:zephyr_net_stack_tx} shows an example of sending a UDP packet.
\begin{enumerate}
        \item The Application sends a chunk of data by using the Socket-API.
        \item The Socket-API binds the request to the context of the opened socket and uses the API to send the data.
        \item The Net-Context implementation calls the IPv6 implementation to fill the IPv6 headers. The interface information, like the source IP address,
              is taken from the interface associated to the Net-Context.
        \item The Net-Context implementation calls the UDP implementation to fill the UDP headers.
        \item The packet is then handed over to the network core.
        \item The network core calls the Network Interface abstraction, which fills the Link-Layer addresses (from either neighborhood discovery or cache).
        \item The Network Interface abstraction then puts the packet into the sending work-queue. The work-queue thread then calls th 6LoCAN Link-Layer.
              The Link-Layer creates a ISO-TP context and performs the fragmentation.
        \item For every frame that is sent, the CAN network driver is called.
\end{enumerate}

\FloatBarrier
\subsection{6LoCAN Network Device Driver}
\label{sec:6locan_net_dev}

The 6LoCAN Network Device Driver is responsible for receiving and sending RAW CAN frames.
For this purpose it is using the Zephyr CAN API.
The API exposes five functions:
\begin{itemize}
        \item Interface Initialization
        \item Enabling/Disabling
        \item Sending
        \item Attaching Filters
        \item Detaching Filters
\end{itemize}

\textbf{Interface Initialization} binds the driver context to an interface, calls the initialization function of the 6LoCAN Link-Layer and registers the multicast monitor.
The multicast monitor is called whenever a multicast group is added to an interface.

\textbf{Enable} is called from the 6LoCAN Link-Layer whenever the interface changes its state from up (enabled) to down (disabled) or vice versa.
If the interface is going to be brought up, then this function attaches the unicast CAN filter.
This filter receives unicast frames where the destination address matches the interface link-local address.
If the interface is going to be brought to the down state, the unicast filter is detached.

\textbf{Send} forwards the incoming frames to the Zephyr CAN API without any modification.

\textbf{Attach Filter} and \textbf{Detach Filter} are wrappers for the Zephyr CAN API, and work the same way the Zephyr CAN API do.
They are used to attach filters for the Link-Layer Duplicate Address Detection (LLDAD), from the 6LoCAN Link-Layer.

Whenever a frame on the bus matches the unicast filter, the receiving function is called.
This function allocates a net packet that is large enough to hold the frame payload, the source and the destination Link-Layer address.
The addresses are taken from the frame identifier and copied into the packet, followed by the frame payload.
The resulting packet is then put into the receiving work-queue of the network stack, which later passes it to the 6LoCAN Link-Layer.

The multicast monitor is responsible for attaching filters for the corresponding multicast group.
The filter matches the the last 14 bits of the IP address and the multicast flag. The source address is ignored for the comparison.
For receiving the frames, the same receiving function is used, as it is used for unicast frames.

\subsection{6LoCAN Link-Layer}
\label{sec:6locan_l2}

The 6LoCAN Link-Layer gets the raw CAN frames from the 6LoCAN Network Device Driver in the thread context of the receiving queue.
It is responsible for handling the ISO-TP transfers that can either be a single frame or a fragmented transfer consisting of many frames.
If the packet does not fit into a single frame, the fragmentation, reassembly and flow control is also managed by this layer.
Fragmented packets are associated with a context that represents a single ISO-TP transfer.
This context contains all the information necessary to handle the segmentation, reassembly and flow control.
The LLDAD is also performed from this layer whenever the interface is initialized.

Whenever a frame arrives, the ISO-TP PCI type is checked to determine what to do next with the frame.

Single frames can be handled in one shot and therefor need no additional context.
The implementation removes the ISO-TP header, checks the length and performs the IPHC decompression.
After that, the packet is processed by the higher layers.

If a First Frame arrives, the implementation reads the total packet length and allocates a new packet that is able to carry the reassembled data.
To keep track of the reassembly process, an ISO-TP context is allocated and linked with the newly allocated packet.
The context contains the actual state of reassembly, timeout timers, a remaining data counter, sequence number and the block counter.
The context is then initialized with the data from the first frame, the residual data is copied to the packet and a if the it is not a multicast transfer,
a Frame Control Frame is sent back to the sender.

For Consecutive frames, the implementation checks if is a context that is in the state of receiving consecutive frames and is linked to a packet from the same sender.
If a context could be found, the data in the Consecutive Frame is added to the packet.
If the counter for remaining data in this context reaches zero, the reassembled packet is pushed to the receiving queue and ands up in this implementation again.
On the finished packet, IPHC decompression is performed and the upper layers continue processing the packet.

Outgoing packets can either be single frames, if they are short enough, or need to be split into several frames.
The frame Identifier is composed from the Link-Layer addresses.
The Link-Layer source and destination address is already defined by the packet.
If the packet can be sent as a single frame, the implementation creates a CAN frame with an ISO-TP Single Frame header and the data from the packet.
If the packet needs several frames, the implementation allocates an ISO-TP sending context, initializes it with the information from the packet,
and sends out the ISO-TP first frame.
The implementation is then waiting for a Frame Control Frame to continue with sending the Consecutive frames.
The implementation sends as many consecutive frames as necessary to complete the transfer.
