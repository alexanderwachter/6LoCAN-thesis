\section*{Abstract}
\label{cha:abstract}

Nowadays, there is an ongoing trend towards end-to-end IPv6 for constrained devices.
This way, the devices can benefit from the vast amount of application and transport layer protocols, defined on top of the Internet Protocol.
Examples for such application layer protocols are MQTT, CoAP, HTTP, or transport layer protocols such as UDP, TCP, and TLS.
However, most devices on the Internet use Link-Layers that do not fit the needs of constrained devices like power consumption, price, or PCB footprint.

The Controller Area Network (CAN) is a very robust and simple bus.
Lots of tiny microcontrollers have an integrated CAN controller that only needs an external transceiver to connect to a bus.
This bus is usually used in the automotive and industrial domains.
An example is the CANopen protocol, designed and used for automation.
However, the protocols for the CAN bus serve a dedicated purpose and are not as flexible as the Internet Protocol.

Therefore, this work proposes 6LoCAN, an abstraction layer for the CAN bus, which combines the great flexibility of IPv6 with the benefits of the CAN bus.
With 6LoCAN, it is possible to connect small microcontrollers to the Internet, with only little effort.
Those devices can then use application layer protocols to communicate with devices that have Link-Layers of all kinds, like Wi-Fi, Ethernet, or Bluetooth, without the need for a gateway.
Meanwhile, 6LoCAN has an IETF standard proposal and a reference implementation in the Zephyr Real-Time Operating System, as an outcome of this work.

\newpage

\section*{Kurzfassung}
Heutzutage erscheinen vermehrt elektonische Kleingeräte die über Ende-zu-Ende IPv6 kommunizieren.
Auf diese Weise profitieren solche Geräte von den vielen Anwender- und Transport-Protokollen,
die auf dem Internet Protokoll basieren.
Beispiele für solche Anwender-Protokollen sind MQTT, CoAP, HTTP oder Transport-Protokolle wie UDP, TCP und TLS.
Nichtsdestotrotz verwenden die meisten Geräte im Internet Verbindungsschichten, die den Anforderungen
wie Energieverbrauch, Preis oder Flächenverbrauch auf der Leiterplatte von Kleingeräten nicht besonders gut erfüllen.

Controller Area Network (CAN) ist ein sehr einfacher und robuster Bus.
Viele kleine und günstige Mikrokontroller haben einen integrierten CAN Modul.
Diese benötigen nur noch einen Bus-Treiber Baustein um sich mit dem Bus zu verbinden.
Der CAN Bus wird hauptsächlich im Automobil und Industrie Sektor verwendet.
Ein Beispiel dafür ist das CANopen Protokoll.
Nichtsdestotrotz sind die verwendeten Protokolle für eine dedizierte Aufgabe geschaffen und verfügen nicht über die Flexibilität des Internet Protokolls.

Deswegen stellt diese Arbeit 6LoCAN vor, einene 6Lo Abstraktionsschicht für den Controller Area Network Bus,
welche die Flexibilität des Internet Protokolls mit den Vorteilen des CAN Bus verbindet.
Mit 6LoCAN ist es möglich einfache Mikrokontroller, mit geringem Aufwand, mit dem Internet zu verbinden.
Diese Geräte können wiederum über die Anwenderschicht-Protokolle mit geräten kommunizieren,
die eventuell ganz andere \\Verbindungsschichten wie Ethernet, Wi-Fi oder Bluetooth verwenden.
Für 6LoCAN existiert mittlerweile ein Vorschlag für einen Standard bei der IETF und eine referenz Implementierung im Zephyr Echtzeitbetriebsystem,
welche im Zuge dieser Arbeit entstanden sind.
