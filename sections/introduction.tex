
\chapter{Introduction}
\label{cha:introduction}

Nowadays, there is an ongoing trend towards end-to-end IPv6 for constrained devices.
This trend is called the Internet Of Things (IoT). It means that many devices are connected,
using internet technology. Some devices like smartphones or personal computers usually have a connection to the Internet out of the box,
but more and more devices that usually don’t have a connection to the Internet become connected devices. Examples are light-bulbs, fridges or
building automation systems \cite{6296937}.
The advantage of using the Internet Protocol is that there are plenty of standardized application layer protocols like CoAP or MQTT,
that can be used for those devices. ”Project Connected Home over IP” \cite{connhome}, for example, is a project from global players like
Apple, Google, Amazon, and many more, that is trying to establish a standardized interface for home application, based on the Internet Protocol.
With this kind of interface, devices from different vendors and different communication mediums, like Wi-Fi, Ethernet, or Bluetooth Low Energy (BLE),
can work together seamlessly.

The Eclipse Sparkplug working group \cite{sparkplug} is currently working on defining a standard to use MQTT in the Industrial IoT.
This initiative shows that there is a high demand for IP technologies in the industry.

There are lots of Link-Layer technologies already in use. They all have their domain-specific use-case.
Ethernet or Wi-Fi are mostly used for high-speed data links suitable for the typical use case of a PC or smartphone.
When using these technologies for tiny devices like a light-switch, problems like high costs, large PCB footprint, or high energy consumption may arise.
The IoT, as we have it today, is mostly based on wireless technologies, but sometimes it is just not feasible to use a wireless link.
Wireless links are prone to electromagnetic disturbances and have problems with range, especially when used in buildings with massive walls.
The CAN-bus is a very robust and cheap bus that is widely used in the automotive and industrial automation (CANopen \cite{canopen}) domain.
It is also used for heating systems and thus already used in building automation today.

The combination of IPv6 and the CAN bus could be very useful to solve lots of challenges with technology we are already using today.
With 6LoCAN, it is possible to connect devices with a wired bus to any other device that works with IPv6.
